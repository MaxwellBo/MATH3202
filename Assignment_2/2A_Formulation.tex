\documentclass[a4paper]{article}
\usepackage{fullpage}
\usepackage{amsmath}
\usepackage{amssymb}

\usepackage[
top=0.3in, % was 0.7
% bottom=1in, % was 0.55
nohead]{geometry}

\title{MATH3202 - Integer Programming - Section A}
\author{Maxwell Bo  \and Chantel Morris}

\begin{document} 

\maketitle

\section*{Sets}

\begin{tabular}{rl}
    $J$ & Set of juices\\
    $F$ & Set of types of fruit\\
    $Q$ & Set of quarters\\
\end{tabular}

\section*{Data}

\begin{tabular}{rl}
    $b_{jf}$ & Fraction (?) $\in [0, 1]$ of fruit $f \in F$ in juice $j \in J$\\
    $c_{f}$ & Cost (\$/kL) of local fruit $f \in F$\\
    $d_{jq}$ & Anticipated ability to sell kL of juice $j \in J$ in quarter $q \in Q$\\
    $b_{q}$ & Demand of kL of orange juice in Brisbane in quarter $q \in Q$\\
    $g_{j}$ & Whether a juice $j \in J$ is gourmet\\
    $s$ & Sell price (\$/kL) of any juice
\end{tabular}

\section*{Variables}

\begin{tabular}{rl}
    $x_{jq}$ & Number of kL of juice $j \in J$ produced in quarter $q \in Q$\\
    $t_{fq}$ & ???
\end{tabular}

\section*{Objectives}

\[
\text{max} \sum_{j \in J}\sum_{q \in Q} (s \cdot x_{jq} -\: (\sum_{f \in F} b_{jf} \cdot c_f) \cdot x_{jq})
\]

\section*{Constraints}

\begin{align}
x_{jq} & \geq 0 & \forall j \in J,\  \forall q \in Q \label{C1}\\
s_{cq} & \geq 0 & \forall c \in C,\  \forall q \in Q \label{C2}\\
\sum_{c \in C} x_{cq} & \leq 10000 & \forall q \in Q \label{C3}\\
i_{c} +  x_{cq} - d_{cq} & = s_{cq} & \forall c \in C,\  \forall q \in \{ f \} \label{C4}\\
s_{c(q - 1)} +  x_{cq} - d_{cq} & = s_{cq} & \forall c \in C,\  \forall q \in Q \setminus \{ f \} \label{C5}\\
s_{cl} & \geq 3000 & \forall c \in C \label{C6}\\
s_{cq} & \leq m_c & \forall c \in C,\  \forall q \in Q \label{C7}
\end{align}

where $f$ is the first quarter, and $l$ is the last quarter, where $f, l \in Q$.
\begin{itemize}
    \item Constraints (\ref{C1}) and (\ref{C2}) are basic non-negativity constraints on our variables. 
    \item Constraints (\ref{C3}) ensures that the amount shipped per quarter does not exceed the ships capacity.
    \item Constraints (\ref{C4}) and (\ref{C5}) describe a recursive relationship between initial supplies, new deliveries, demand, and the amount stored at the end of each quarter.
    \item Constarint (\ref{C6}) ensures that there at least 3000 barrels in storage in each port by the end of the last quarter.
    \item Constarint (\ref{C7}) ensures that we do not exceed the capacities of our facilities in each port.
\end{itemize}
\end{document}